\documentclass{article}

\usepackage{fullpage}
\usepackage{color}
\usepackage{amsmath}
\usepackage{url}
\usepackage{verbatim}
\usepackage{graphicx}
\usepackage{parskip}
\usepackage{amssymb}
\usepackage{nicefrac}
\usepackage{listings} % For displaying code

% Answers
\def\ans#1{\\\gre{Answer: #1}}

% Colors
\definecolor{blu}{rgb}{0,0,1}
\def\blu#1{{\color{blu}#1}}
\definecolor{gre}{rgb}{0,.5,0}
\def\gre#1{{\color{gre}#1}}
\definecolor{red}{rgb}{1,0,0}
\def\red#1{{\color{red}#1}}
\def\norm#1{\|#1\|}

% Math
\def\R{\mathbb{R}}
\def\argmax{\mathop{\rm arg\,max}}
\def\argmin{\mathop{\rm arg\,min}}
\newcommand{\mat}[1]{\begin{bmatrix}#1\end{bmatrix}}
\newcommand{\alignStar}[1]{\begin{align*}#1\end{align*}}

% LaTeX
\newcommand{\fig}[2]{\includegraphics[width=#1\textwidth]{a0f/#2}}
\newcommand{\centerfig}[2]{\begin{center}\includegraphics[width=#1\textwidth]{#2}\end{center}}
\def\items#1{\begin{itemize}#1\end{itemize}}
\def\enum#1{\begin{enumerate}#1\end{enumerate}}



\begin{document}



\title{CPSC 340 Assignment 0 Solutions}
% \author{Warm-up}
\date{}
\maketitle
\vspace{-4em}

\textit{***Collaborated with Zack Wong on this assignment***}
\section{Linear Algebra Review}

\subsection{Basic Operations}

\enum{
\item 14
\item 0
\item $\left[\begin{array}{c}
6\\
10\\
14\\
\end{array}\right]$
\item $\sqrt{5}$
\item $\left[0, 1, 2\right]$
\item $
\left[\begin{array}{ccc}
	3 & 1 & 1\\
	2 & 3 & 1\\
	2 & 1 & 3
\end{array}\right]$
\item $\left[\begin{array}{c}
6\\
5\\
7\\	
\end{array}\right]$
}

\subsection{Matrix Algebra Rule}

\begin{enumerate}
\item True
\item True
\item True
\item False
\item False
\item True
\item False
\item True
\item True
\end{enumerate}

\subsection{Special Matrices}

\enum{
\item Symmetric matrix: a matrix that is equal to its transposed matrix
\item Identity matrix: a square matrix that has 1 for all the diagonal entries and 0 for the rest of the entries
\item Orthogonal matrix: a matrix the transpose of which is equal to its inverse
}



\section{Probability Review}



\subsection{Rules of probability}

\begin{enumerate}
\item $\frac{1}{4}$
\item \$4
\item 0.55
\end{enumerate}


\subsection{Bayes Rule and Conditional Probability}

\begin{enumerate}
\item 0.010094
\item True positive because $P(T=1|D=1)$ and $(T=0|D=0)$ have high probabilities
\item 0.009411531
\item No, $P(D=1|T=1)$ should be higher if the tests are true positive.
\item $P(T=1|D=1)$ needs to be higher (the test should be more accurate)
\end{enumerate}

\subsection{Bayes Rule and Independence}

\textit{***Sources used: https://en.wikipedia.org/wiki/Monty\_Hall\_problem}

There are 3 possibilities:
\begin{enumerate}
	\item $PEE$
	\item $EPE$
	\item $EEP$\\
\end{enumerate}

\quad *$P$ = Prize, $E$ = Empty\\

Given that we choose door 1,
\begin{itemize}
	\item If case 1 is true, then switching is disadvantageous, and not switching advantageous
	\item If case 2 is true, then switching is advantageous, and not switching disadvantageous
	\item If case 3 is true, then switching is advantageous, and not switching disadvantageous
\end{itemize}

Each case has the possibility of $\frac{1}{3}$ of occuring, so there is a $\frac{2}{3}$ chance that switching is advantageous vs. $\frac{1}{3}$ that not switching is advantageous. So, the contestant should b) switch to door 2.

\section{Calculus Review}

\subsection{One-variable derivatives}

\begin{enumerate}
\item $\frac{14}{3}$
\item $\frac{1}{4}$
\item 0
\item $-\frac{1}{e^x+1}$
\end{enumerate}


\subsection{Multi-variable derivatives}

\begin{enumerate}
% \item $f_1(x) = \sin(x)$
\item $\langle2x_1,e^{x_2}\rangle$.
\item $\langle e^{x_1+x_2x_3}, x_3e^{x_1+x_2x_3}, x_2e^{x_1+x_2x_3}\rangle$.
\item $\vec{a}$.
\item $\langle4x_1-2x_2, 2x_2-2x_1\rangle$.
\item $\vec{x}$.
\end{enumerate}

Hint: it is helpful to write out the linear algebra expressions in terms of summations.



\subsection{Derivatives of code}

\begin{enumerate}
\item
	\begin{verbatim}
		function grad1(x)
        	n = length(x);
        	g = zeros(n);
        	for i in 1:n
                g[i] = 3x[i]^2;
        	end
        	return g
		end
	\end{verbatim}
\item
	\begin{verbatim}
		function grad2(x)
        	n = length(x);
        	g = ones(n);
        	for i in 1:n
                g[i] = prod(x[1:end .!=i]);
        	end
        	return g
		end
	\end{verbatim}
\item
	\begin{verbatim}
		function grad3(x)
        	n = length(x);
        	g = zeros(n);
        	for i in 1:n
                g[i] = 1 / (exp(x[i]) + 1);
        	end
        	return g
		end
	\end{verbatim}
\end{enumerate}


\section{Algorithms and Data Structures Review}

\subsection{Trees}

Assuming the depth of a single-node binary tree has a depth of 0:
\begin{enumerate}
\item $2^l$
\item $2^l-1$
\end{enumerate}

\subsection{Common Runtimes}

\begin{enumerate}
\item $O(nlgn)$
\item $O(n)$
\item $O(n)$ (worst case: list contains all 0's, i.e. have to check every element to see if it's 0)
\item $O(nd)$
\item $O(z)$
\end{enumerate}

\subsection{Running times of code}

Assuming \texttt{ones(n, 1)} and \texttt{zeros(n, 1)} are in $O(n)$:
\begin{enumerate}
	\item \texttt{func1} $\in O(n)$
	\item \texttt{func2} $\in O(1)$
	\item \texttt{func3} $\in O(n)$
	\item \texttt{func4} $\in O(n^2)$
\end{enumerate}
\end{document}